\documentclass[12pt]{article}
\usepackage{graphicx}
\usepackage{amsmath}
\usepackage{hyperref}

\title{Forecasting US Inflation}
\author{Udacity Data Scientist Nanodegree Capstone project}
\date{\today}

\begin{document}
\maketitle
\tableofcontents
\newpage

\section{Project Definition}
\subsection{Project Overview}
state the high-level overview of the project, including the background information such as problem domain, project origin, and related data sets or input data.
\subsection{Problem Statement}
define the problem to be solved.
\subsection{Metrics}
define the metrics to measure the results and justifications to use the metrics. For example, if you use time-series data sets, what metrics will be appropriate to measure the results.

\section{Analysis}
\subsection{Data Exploration}
describe the data sets, including the features, data distributions, and descriptive statistics. Identify any abnormalities or specific characteristics inherent in the data sets.
\subsection{Data Visualization}
 build data visualization based on the data exploration in the 

\section{Methodology}
\subsection{Data Preprocessing}
describe the steps taken to preprocess the data and address any abnormalities in the data sets. If data preprocessing is not needed, please explain why.
\subsection{Implementation}
discuss the process using the models, algorithms, and techniques applied to solve the problem. Any complications during the implementation should be mentioned.
\subsection{Refinement}
describe the process to refine the algorithms and techniques, such as using cross-validation or changing the parameter settings.

\section{Results}
\subsection{Model Evaluation and Validation}
discuss the models and parameters used in the methodology. If no model is used, students can discuss the methodology using data visualizations and other means.
\subsection{Justification}
 discuss the final results in detail and explain why some models, parameters, or techniques perform better over others. Show and compare the results in tabular forms or charts.

\section{Conclusion}
\subsection{Reflection}
 summarize the end-to-end problem solution and discuss one or two particular aspects that you find interesting or difficult to implement.
\subsection{Improvement}
 provide suggestions for the next research to improve the experiment.



\end{document}
