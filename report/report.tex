\documentclass[12pt]{article}
\usepackage{graphicx}
\usepackage{svg}
\usepackage{amsmath}
\usepackage{hyperref}
\usepackage{booktabs} 
\usepackage[showframe=false]{geometry}
\usepackage{changepage}
\usepackage{longtable}

\title{Forecasting US Inflation}
\author{Udacity Data Scientist Nanodegree Capstone project}
\date{\today}

\begin{document}
\maketitle
\tableofcontents
\newpage

\section{Project Definition}
\subsection{Project Overview}

This is the Udacity Data Scientist Nanodegree Capstone project. The aim is to demonstrate the concepts learned in the programme. To do this, I have set up a project that includes
\begin{itemize}
\item Collecting the data (web scrap and using APIs)
\item Cleaning and storing the data in a database
\item Writing modular, documented code 
\item Analysing different learning algorithms
\item Drawing conclusions and communicating them
\end{itemize}

Because of my background, I decided to work on an economic problem, namely forecasting inflation. Inflation measures the general increase in the price of goods and services over time. Inflation affects the purchasing power of consumers and is an important economic variable. Inflation forecasting is particularly relevant in the current context of high inflation rates following the COVID pandemic.  The current discussion is whether and how fast inflation will return to pre-COVID levels. 

The aim of this project is to analyse different inflation forecasting methods and to compare their accuracy and reliability. The project will focus on the US case. Macroeconomic variables will be retrieved from the Federal Reserve Economic Data API and by webscraping the BLS homepage.


\subsection{Problem Statement}


The main research question of this project is: Which inflation forecast method is the most accurate and reliable? What is the current prediction in the high inflation environment?

To answer this we compare different inflation forecast methods for the US case. We compare the forecasting power among the following dimensions: 
\begin{itemize}
\item Is it helpful to forecast the components of the consumer price index individually and aggregating them?
\item Do time series models gain forecasting power by adding economic variables (macroeconomic variables, such as output, unemployment, and exchange rates. )?
\item Does it help to use non-linear models?
\item How do the model predictions compare to expert forecasts (survey data)?
\end{itemize}

\subsection{Metrics}
The main metric for evaluating the inflation forecast methods is the root mean squared error (RMSE), which measures the average deviation of the forecasted values from the actual values. The lower the RMSE, the better the forecast method.
The RMSE of a naive forecast method (such as a random walk or a no-change forecast) will be shown as benchmark.
Apart from the RMSE we look (out of sample in the test data) at the sample mean. As I will explain later on, it makes the exercise more useful to split the data into test and train using a timestamp (and not using random shuffling). It is not guaranteed that out of sample the error is zero on average. Therefore, looking at the average error is also useful. 

\section{Analysis}
\subsection{Data Exploration}

Table \ref{table:overview} shows the subcomponents of the CPI (the parts the CPI) is made of together with the date when the item appeared on the CPI and the distribution of the monthly (seasonally adjusted, non annualized) CPI changes. 
The "All items" CPI component  (the one we want eventually to forecasts) is available since 1947. On average the monthly increase was 0.29\% but the increases range from -1.77\% to 1.96\%.
We see that while some items are available for the whole period (like Food) others are only available since 1994 (video and audio) or even 1999 (personal care). This reflects the fact that products which are available to the public change over time. They do not only change the weight in the CPI but completely new products may emerge over time. 



The most volatile components of the CPI are: 
\begin{itemize}
\item Tobacco and smoking products
\item Private and public transportation
\item Fuels and utilities 
\item Food
\item Apparel (mens and boys, womens and girls, infants)
\end{itemize}


The least volatile components of the CPI are: 
\begin{itemize}
\item Education
\item Medical care
\item Personal care
\item Housing
\end{itemize}

Table \ref{table:weights} shows the weights that the subcomponents (introduced in table \ref{table:overview}) do have in the CPI for some selected years. We do omit most of the years to save space. 
Observations:
\begin{itemize}
\item The most important component has always been Housing and the weight has increased  from 30\% to 40\% over time. 
\item Food is also an important part of the CPI but its importance has decreased from 30\% to around 15\% over time. 
\item The next important category is private transportation which has increased from 10\% to 16\% over time. 
\end{itemize}


Table \ref{table:motivation} shows the additional variables used in the analysis which are not the CPI compents itself. The additional variables are motivated by economic reasoning and involve concepts that can be reasonably expected to have an impact on price levels. The variables include, among others:
\begin{itemize}
\item Unemployment rates
\item Loan data
\item Money supply (money aggregates)
\item Interest rates
\item Foreign exchange rates
\end{itemize}
The motivation to use the variables is summarized in the table.


 \begin{adjustwidth}{-2cm}{}
\begin{table}
\begin{tabular}{llrrrr}
{} & {first date} & {mean} & {std} & {min} & {max} \\
{FRED Name} & {} & {} & {} & {} & {} \\
\hspace*{0ex}All items & 1947-01-01 & 0.29\% & 0.34\% & -1.77\% & 1.96\% \\
\hspace*{4ex}Food and beverages & 1967-01-01 & 0.33\% & 0.41\% & -0.96\% & 5.17\% \\
\hspace*{8ex}Food & 1947-01-01 & 0.29\% & 0.56\% & -2.32\% & 5.87\% \\
\hspace*{4ex}Housing & 1967-01-01 & 0.35\% & 0.31\% & -0.81\% & 1.86\% \\
\hspace*{8ex}Shelter & 1953-01-01 & 0.34\% & 0.35\% & -1.43\% & 2.24\% \\
\hspace*{8ex}Fuels and utilities & 1953-01-01 & 0.31\% & 0.73\% & -2.40\% & 5.61\% \\
\hspace*{8ex}Household furnishings and operations & 1967-01-01 & 0.19\% & 0.33\% & -0.64\% & 2.02\% \\
\hspace*{4ex}Apparel & 1947-01-01 & 0.13\% & 0.47\% & -3.66\% & 1.87\% \\
\hspace*{8ex}Mens and boys apparel & 1947-01-01 & 0.13\% & 0.58\% & -4.42\% & 2.61\% \\
\hspace*{8ex}Womens and girls apparel & 1947-01-01 & 0.08\% & 0.75\% & -4.09\% & 2.92\% \\
\hspace*{8ex}Footwear & 1947-01-01 & 0.19\% & 0.57\% & -2.58\% & 3.10\% \\
\hspace*{8ex}Infants and toddlers apparel & 1989-01-01 & 0.02\% & 1.39\% & -4.13\% & 9.57\% \\
\hspace*{4ex}Transportation & 1947-01-01 & 0.30\% & 1.03\% & -10.28\% & 5.87\% \\
\hspace*{8ex}Private transportation & 1947-01-01 & 0.29\% & 1.07\% & -10.80\% & 6.22\% \\
\hspace*{8ex}Public transportation & 1989-01-01 & 0.19\% & 1.80\% & -11.62\% & 11.09\% \\
\hspace*{4ex}Medical care & 1947-01-01 & 0.41\% & 0.31\% & -0.68\% & 1.82\% \\
\hspace*{8ex}Medical care commodities & 1967-01-01 & 0.32\% & 0.34\% & -0.85\% & 1.45\% \\
\hspace*{8ex}Medical care services & 1956-01-01 & 0.45\% & 0.31\% & -0.70\% & 2.12\% \\
\hspace*{4ex}Recreation & 1993-01-01 & 0.11\% & 0.23\% & -0.60\% & 0.87\% \\
\hspace*{8ex}Video and audio & 1994-01-01 & 0.05\% & 0.33\% & -1.25\% & 1.16\% \\
\hspace*{4ex}Education and communication & 1993-01-01 & 0.15\% & 0.23\% & -1.72\% & 1.12\% \\
\hspace*{8ex}Education & 1993-01-01 & 0.37\% & 0.17\% & -0.24\% & 1.19\% \\
\hspace*{8ex}Communication & 1998-01-01 & -0.09\% & 0.41\% & -3.31\% & 1.81\% \\
\hspace*{4ex}Other goods and services & 1967-01-01 & 0.41\% & 0.40\% & -0.98\% & 4.07\% \\
\hspace*{8ex}Tobacco and smoking products & 1986-01-01 & 0.56\% & 1.60\% & -5.19\% & 17.74\% \\
\hspace*{8ex}Personal care & 1999-01-01 & 0.19\% & 0.21\% & -0.28\% & 1.16\% \\
\end{tabular}



\caption{Overview of the used CPI items}
\label{table:overview}
\end{table}
\end{adjustwidth}



 \begin{adjustwidth}{-2cm}{}
\begin{table}


\begin{tabular}{lllll}
{} & {1952} & {1972} & {1992} & {2012} \\
{FRED Name} & {} & {} & {} & {} \\
\hspace*{0ex}All items & 100.00\% & 100.00\% & 100.00\% & 100.00\% \\
\hspace*{4ex}Food and beverages & NA & NA & 17.40\% & 15.26\% \\
\hspace*{8ex}Food & 29.84\% & 22.49\% & 15.78\% & 14.31\% \\
\hspace*{4ex}Housing & 32.18\% & 33.86\% & 41.40\% & 41.02\% \\
\hspace*{8ex}Shelter & 17.46\% & 21.83\% & 27.88\% & 31.68\% \\
\hspace*{8ex}Fuels and utilities & NA & 4.71\% & 7.28\% & 5.30\% \\
\hspace*{8ex}Household furnishings and operations & 6.45\% & 7.32\% & 6.24\% & 4.04\% \\
\hspace*{4ex}Apparel & 9.42\% & 10.37\% & 6.00\% & 3.56\% \\
\hspace*{8ex}Mens and boys apparel & 3.00\% & 2.80\% & 1.42\% & 0.86\% \\
\hspace*{8ex}Womens and girls apparel & 4.16\% & 3.98\% & 2.46\% & 1.50\% \\
\hspace*{8ex}Footwear & 1.44\% & 1.57\% & 0.80\% & 0.70\% \\
\hspace*{8ex}Infants and toddlers apparel & NA & NA & 0.19\% & 0.20\% \\
\hspace*{4ex}Transportation & 11.33\% & 13.13\% & 17.01\% & 16.85\% \\
\hspace*{8ex}Private transportation & 10.11\% & 11.66\% & 15.48\% & 15.66\% \\
\hspace*{8ex}Public transportation & 1.22\% & 1.47\% & 1.53\% & 1.19\% \\
\hspace*{4ex}Medical care & 4.78\% & 6.45\% & 6.93\% & 7.16\% \\
\hspace*{8ex}Medical care commodities & NA & NA & 1.28\% & 1.71\% \\
\hspace*{8ex}Medical care services & 3.99\% & 5.58\% & 5.65\% & 5.45\% \\
\hspace*{4ex}Recreation & NA & 3.77\% & NA & 5.99\% \\
\hspace*{8ex}Video and audio & NA & NA & NA & 1.90\% \\
\hspace*{4ex}Education and communication & NA & NA & NA & 6.78\% \\
\hspace*{8ex}Education & NA & NA & NA & 3.28\% \\
\hspace*{8ex}Communication & NA & NA & NA & 3.50\% \\
\hspace*{4ex}Other goods and services & 5.01\% & 5.09\% & 6.90\% & 3.38\% \\
\hspace*{8ex}Tobacco and smoking products & NA & NA & 1.75\% & 0.80\% \\
\hspace*{8ex}Personal care & 2.12\% & 2.57\% & 1.19\% & 2.57\% \\
\end{tabular}


\caption{Weights of the CPI items in some selected years. Not all weights are available at all time.}
\label{table:weights}
\end{table}
\end{adjustwidth}


 \begin{adjustwidth}{-2cm}{}
\begin{longtable}{llp{6cm}}

\toprule
{} &          available since & motivation \\
\midrule
WTI                              & 1946-01-31 & The oil price is an important driver of the overall price level of the economy. A higher oil price contemporeanously affects CPI. It can be used to nowcast the inflation. Note, however, that the oil price can only be used to forecast CPI if it affects certain price categories with a *lag*.  \\
Consumer\_Loans                       & 1947-01-31 & A decrease in consumer loans can lead to a decrease in consumer spending, which can cause a decrease in demand for goods and services.  \\
Loans\_Leases                         & 1947-01-31 & Similar to consumer loans, but here also loans to enterprises are included.\\
Unemployment\_Rate                       & 1948-01-31 & When unemployment is high, there’s little need for employers to “bid” for the services of employees by paying them higher wages. In times of high unemployment, wages typically remain stagnant, and wage inflation is non-existent. \\
10Y\_Rate                                     & 1953-04-30 & Interest rate on government debt with 10 year maturity. When interest rates are high, borrowing becomes more expensive, and people tend to spend less money. This decrease in spending can lead to a decrease in demand for goods and services, which can lead to a decrease in prices and a decrease in the CPI . \\
FFER                                     & 1954-07-31 &  Fed Funds effective rate: Interest rate on short term loans between banks. Same motiviation as for 10 Y rates. \\

Real\_M1                             & 1959-01-31 & M1, M2, and M3 are monetary aggregates that represent different measures of the money supply in an economy. M1 is the narrowest measure of the money supply and includes currency, demand deposits, and other liquid assets. M2 and M3 include other liquid assets.  (An increase in the money supply (M1 to M3) can lead to inflation, which can cause the CPI to rise.)\\
M1                                     & 1959-01-31 & see Real\_M1\\
M2                           & 1959-01-31  & see Real\_M1 \\
Real\_M2                       & 1959-01-31 & see Real\_M1 \\
M3                                    & 1960-01-31  & see Real\_M1\\
JPY                                 & 1971-01-31 &  Exchange rate (Japan), import prices may rise, triggering increases in the CPI of the home country. \\
CAD                               & 1971-01-31 & Exchange rate (Canada), see JPY for motivation \\
GBP                                   & 1971-01-31  & Exchange rate (United Kingdom), see JPY for motivation \\
Real\_Borad\_Effective\_Exchange          & 1994-01-31 & Exchange rate (aggregate, real), see JPY for motivation \\
Inflation\_Exp\_Market\_10YR & 1982-01-31 & Market expectation of inflation (10Y expectatin) derived from prices of inflation protected bonds. \\
Inflation\_Exp\_Market\_1YR & 1982-01-31  & Market expectation of inflation (1Y expectatin) derived from prices of inflation protected bonds. \\

\bottomrule



\caption{Additional variables used to forecast CPI}
\label{table:motivation}
\end{longtable}
\end{adjustwidth}


\subsection{Data Visualization}


\begin{figure}[h]
    \centering
    \includegraphics[width=1\textwidth]{cpi_vs_target_yoy.pdf}
    \caption{Year over Year changes of CPI (seasonally adjusted) vs target}
    \label{fig:cpi_vs_target_yoy}
\end{figure}


\begin{figure}[h]
    \centering
    \includegraphics[width=1\textwidth]{cpi_vs_target_mom.pdf}
    \caption{annualized month over over month changes of CPI vs target}
    \label{fig:cpi_vs_target_mom}
\end{figure}

\begin{figure}[h]
    \centering
    \includegraphics[width=1\textwidth]{weights.pdf}
    \caption{Weights in the cpi basket}
    \label{fig:weights}
\end{figure}

Figure \ref{fig:cpi_vs_target_yoy} shows how the CPI increases over time. The inflation target of the FED is 2\% p.a.. We see that the FED has missed the target in the long run. However, this is due to certain periods where inflation overshoots (like 1970-1984) or the most recent episode. There has been longer periods where the inflation moves in line with the target (1989-2020). It is important to note that the CPI is often commented in year over year (YoY) changes. However, YoY are easy to forecast before the data realease because before the releases 11 out of the 12 months that make a YoY datapoint are already known (so called based effects). It is much harder to forecast the month over month (MoM) changes that are shown in figure  \ref{fig:cpi_vs_target_mom}. In the following we are going to forecast YoY changes of the CPI but with a forecasting horizon of 12 months. This means we are forecasting YoY inflation in one years time and so we do not make the exercise easier by incorporating base effects. 
Figure \ref{fig:weights} shows the weights of the items in the CPI basket over time. As already pointed out shelder is one of the most important compents. Note also that the weights do not sum up to 100\%. This is due to missing data and to the fact that some items that are part of the CPI today did not exist 40 years ago (i.e. Video and audio). Missing data is one topic to be treated separately. 


\section{Methodology}
\subsection{Data Preprocessing}

The FRED API requires registering and getting an API key. Registering at FRED is free and you get an API key within minutes. However, we should never hardcode authentication credentials like tokens, keys, or app-related secrets into code published to Github.  
I checked several ways to solve the problem here: \href{https://docs.github.com/en/rest/overview/keeping-your-api-credentials-secure?apiVersion=2022-11-28}{Keeping your API credentials secure (Github post)}. I decided to use the Python dotenv package, because it is a straightforward solution. Instructions on how to write the API key into the environment is provided in the submission comment for the project. 


The overall goal of the CPI-U index is to use consumer spending from as recent a period as possible, and hold the set (or more precisely, the quantity mix) of goods and services purchased fixed over time until new spending weights can be introduced. In general, estimates of current period inflation calculated with outdated spending weights tend to be higher than inflation estimates calculated with more current spending weights. This is because consumers change, or substitute, what they buy over time, often shifting purchases away from items that are becoming relatively more expensive to alternatives whose prices are not rising as fast.


\begin{figure}[h]
    \centering
    \includegraphics[width=1\textwidth]{cpi_weights_example.png}
    \caption{Example of the subcompents of the CPI}
    \label{fig:cpisub}
\end{figure}

The CPI weights are unfortunately not available directly through an API as far as I have seen. The weights can however be downloaded by scraping the website of the U.S. Bureau of Labor Statistics (BLS). Building the historical weights is however not straightforward, as the names change slightly over time. 
The subbaskets of the CPI are aggregated to categories with different granularity levels. As shown in Figure \ref{fig:cpisub}, the "food and beverages" component (identation level 1) is further divided into "Food" (identation level 2) which can be furtther divided into "Food at home" (level 3)..
Combining the weights (scraped from the U.S. Labor Statistics webpage) with the Index series downloaded from FRED requires some extra effort, because 1) the names do not match exaclty  2) not all CPI baskets reported on the BLS webpage can be downloaded from FRED. 
To match the data I followed the following steps:
\begin{itemize}
\item Download all series available from FRED until identation level 2 (i.e. "Food at home" in figure \ref{fig:cpisub})
\item Match them by name with the weighting data from BLS. In cases where the matching is not possible (because of slightly differnt spelling) a manual mapping is used. Construction of a manual mapping is necessary since the names of the items change over time. 
For instance the item "fuels and utilities" has over 4 different names over time ("Fuels and utilities", "HOUSEHOLD FUELS AND OTHER UTILITIES", "Fuel and other utilities" and "Fuel and utilities")
\end{itemize}
The weighting data from the BLS war hard to clean. The history does not come through an API but rather requires the download of Excel, ZIP and text files. The format of the data has changed 5 times in total:
\begin{itemize}
\item The data between 1952 and 1986 comes in one Excel File
\item Between 1952 and 1981 the data comes  in a sheet of an Excel (several years in one sheet, 1-2 sheets per decade).
\item Between 1982 and 1986 the data comes in several sheets (one sheet per year).
\item Between 1987 and 2018  the data comes in individual txt files (one file per year). The format has changed considerably over time and is not machine friendly. The relevant items need to be extracted with regular expressions.
\item Between 2018 and 2022 the data comes in individual txt files (one file per year). 
\end{itemize}


\subsection{Implementation}
I decide to split the data as follows:
\begin{itemize}
\item Train set: until December 2014
\item Test set: January 2015 - now
\end{itemize}
Note, that in computer science test and train sets are often computed by random splits. Here I do not want to have this approach since datapoints are autocorrelated. I fear that if the model sees part of the Corona inflation schock it fits to it. 


The dataset has a lot of missing values as discussed in the previous sections. The standard approach is to drop the data so that we have a dataset without missing values. However, this simple approach is not suitable here. By dropping the data we miss periods which are very interesting from an economical point of view (like the inflation period in the 1970s). Moreover, the missing data should not be a big problem since it often involes only items that have a low weight in the CPI basket (like audio). I therefore decided to fill the missing missing values as follows:
\begin{itemize}
\item Weight of the items in the CPI basket: Fill with 0 (realistic since the items simply did not exist). Rescale the weights so that they sum up to one (see figure \ref{fig:weights})
\item CPI items: Fill with the value of the overall CPI. This means that the assumption is that an item simply moved in line with the overall price level if missing/non-existent.
\item Other Economic data: Fill with the insample mean of the data. Here I make sure not to use data that belongs to the test set (i.e. data after January 2015).
\end{itemize}

As we see in table \ref{table:motivation} some variables are in percentage points and can be assumed to be stationary, whereas other variables come in Levels (i.e. exchange rates). For variables that come in levels we take pct differences before passing them to the model. It does not make any sense to forecast the inflation using the absolute level of i.e. the exchange rate GBP/USD. It is more realistic to assume that the pct change in the exchange rate (i.e. the USD has gained 4\% against GBP) does influence CPI.

\subsubsection{Benchmark models}
I start with simple (naive) models that can be used as a benchmark. Figure \ref{fig:cb_model} shows the out of sample results of a model that simply uses 2\% as inflation forecast. 2\% is the inflation forecast that the FED uses as long term target.
 Figure \ref{fig:naive_model} shows the out of sample results of a model that simply uses the historical mean as a forecast. Since the historical mean has been higher than 2\% and the covid shock has led to inflation shooting over 2\%, the mean model is doing better than the 2\% model. Both models have the same RMSE (since they predict constant values) but the mean model has a lower average error.


\begin{figure}[h]
    \centering
    \includegraphics[width=1\textwidth]{centralbanktargetmodel.pdf}
    \caption{Out of sample precdiction of a benchmark model that predicts inflation to be at the central bank target (2\%)}
    \label{fig:cb_model}
\end{figure}



\begin{figure}[h]
    \centering
    \includegraphics[width=1\textwidth]{naivemodel.pdf}
    \caption{Out of sample precdiction of a benchmark model that predicts inflation to be at the historical (-2014) mean}
    \label{fig:naive_model}
\end{figure}


\subsubsection{Univariate time series models}
\label{sec:univts}
The next models I would like to test is to forecast inflation as follows:
\begin{equation}
\label{eq:ma}
CPI_{YoY, t, t+12} = \alpha + \beta \cdot MA(CPI_{MoM,t-6,t})+\epsilon
\end{equation}

I.e. those models look predict inflation using an historical average and a moving average (the average of the MoM changes over the past 6 months) to predict future inflation. I estimate the model using OLS and I get the result shown in \ref{fig:univariatetimeseriesmodelma}. The model has a better out of sample RMSE than the naive ones. The beta coefficient is estimated to be positive. This is interesting and means that the model does not capture mean reversion features of inflation but rather predicts high past inflation to be a sign of high inflation in the future. We see this also clearly in the figure \ref{fig:univariatetimeseriesmodelma}.
\begin{figure}[h]
    \centering
    \includegraphics[width=1\textwidth]{univariatetimeseriesmodelma.pdf}
    \caption{Out of sample precdiction of a model that uses the 6 month realized inflation to forecast the future inflation.}
    \label{fig:univariatetimeseriesmodelma}
\end{figure}

An extension of the model would be to use several moving averages (like the 2 and 6 month average). The motivation is that we may want the model to react to short term CPI developments quickly. 
\begin{equation}
CPI_{YoY, t, t+12} = \alpha + \beta_1 \cdot MA(CPI_{MoM,t-6,t})+ \beta_2 \cdot MA(CPI_{MoM,t-2,t})+\epsilon
\end{equation}

Figure \ref{fig:univariatetimeseriesmodelma2and6} shows the results of such a model. The RMSE is lowered compared to the model in figure \ref{fig:univariatetimeseriesmodelma}. 

\begin{figure}[h]
    \centering
    \includegraphics[width=1\textwidth]{univariatetimeseriesmodelma2and6.pdf}
    \caption{Out of sample precdiction of a model that uses the 2 and 6 month realized inflation to forecast the future inflation.}
    \label{fig:univariatetimeseriesmodelma2and6}
\end{figure}
In section \ref{sec:Refinement} we will use a cross-validation approach to find the best moving averages.

\subsubsection{Bottom up aggregation models}
\label{sec:agg}
In this section, I use models that estimate each component of the CPI separately and then aggregate them up using CPI weights. The motivation is that some parts of the CPI may behave differently than other (i.e. being more or less sticky/mean reverting). 
Each component of the CPI is modelled as in equation \ref{eq:ma}. 
\begin{figure}[h]
    \centering
    \includegraphics[width=1\textwidth]{bottomupaggregationmodelma6forindividualvars.pdf}
    \caption{Out of sample precdiction of a model that predicts inflation components individually and sums them up using basket weights.}
    \label{fig:bottomupaggregationmodelma6forindividualvars}
\end{figure}

The results are shown in \ref{fig:bottomupaggregationmodelma6forindividualvars}. Interestingly, the model results in worse prediction than the simple model shown in \ref{fig:univariatetimeseriesmodelma}. So it seems that it is not worth forecasting each variable indepedently.  In table \ref{table:forecast_by_item_r2} I show the in sample R2 that I get for forecasting the individual compoentns, as well as the average weight they have in the out of sample period.
We see huge differences in forecasting power of individual components in the CPI. I.e. Food can be forecasted very poorly and makes 14.4\% of the CPI. Also private transportation cannot be forecasted accurately, but makes 15.9\% of the CPI. 
In section  \ref{sec:Refinement} I will test if it makes sense to just forecast some parts of the CPI and leave the ones that cannot be forecasted. 

\begin{adjustwidth}{-2cm}{}
\begin{table}

\begin{tabular}{lrr}
{} & {R2} & {avg. weight (2015-2023)} \\
Private transportation & 0.7\% & 15.9\% \\
Public transportation & 1.5\% & 1.1\% \\
Womens and girls apparel & 4.3\% & 1.2\% \\
Footwear & 9.3\% & 0.7\% \\
Fuels and utilities & 12.6\% & 4.9\% \\
Tobacco and smoking products & 14.0\% & 0.6\% \\
Infants and toddlers apparel & 15.8\% & 0.1\% \\
Mens and boys apparel & 18.6\% & 0.7\% \\
Food & 19.3\% & 14.4\% \\
Household furnishings and operations & 40.0\% & 4.7\% \\
Shelter & 48.3\% & 35.1\% \\
Education & 49.7\% & 3.1\% \\
Personal care & 51.8\% & 2.6\% \\
Medical care services & 53.6\% & 7.3\% \\
Medical care commodities & 54.4\% & 1.8\% \\
Video and audio & 56.3\% & 1.7\% \\
Communication & 60.8\% & 3.9\% \\
\end{tabular}


\caption{Forcasting power on the indivual CPI items, as well as average weight.}
\label{table:forecast_by_item_r2}
\end{table}
\end{adjustwidth}


\subsubsection{Using additional economic variables}
\label{sec:eco}
In this section I estimate the model presented in \ref{fig:univariatetimeseriesmodelma} but I add as independent variables the additional economic variables introduced in \ref{table:motivation}.
\begin{figure}[h]
    \centering
    \includegraphics[width=1\textwidth]{ma6modelwithaddeconomicvariables.pdf}
    \caption{Out of sample precdiction of a model that uses the 6 month realized inflation and the variables outlined in table \ref{table:motivation} to forecast the future inflation.}
    \label{fig:ma6modelwithaddvars}
\end{figure}
The results are shown in \ref{fig:ma6modelwithaddvars}. The model shows an improvement relative to the simple MA models outlined above. We will see in the section \ref{sec:Refinement} if it makes sense to remove some variables. 

\subsection{Refinement}
\label{sec:Refinement}

Until now we have seen the following results. 
\begin{itemize}
\item Moving average models that use 2-6 month realized inflation to forecast future inflation work reasonably well.
\item Bottom up aggregation models do not work. The components of the CPI basket differ in their difficulty to forecast.
\item Using additional economic variables shows promising results. However, until now we are using too much of them.
\end{itemize}

In this section I use cross-validation techniques to fine tune the parameters. The parameters are fitted on the training data (i.e. up to 2014).
Regarding the models in section\ref{sec:univts}, I investigate if other moving averages do better jobs. I test all the the 1-18 month lags and up to 3 moving averages. So I test (using a grid search) 833 parameter combinations. The best model is one that uses the 1 and the 7 month moving average. It is shown in figure \ref{fig:ma17modelfoundusingcrossvalidation}. The RMSE is reduced further compared to the previous models, althouhg the difference is not large.
\begin{figure}[h]
    \centering
    \includegraphics[width=1\textwidth]{ma17modelfoundusingcrossvalidation.pdf}
    \caption{Out of sample precdiction of a model that uses the 1 and 7 month realized inflation to forecast the future inflation. This is the best parametrization found using cross validation.}
    \label{fig:ma17modelfoundusingcrossvalidation}
\end{figure}


Regarding the bottom up aggregation models presentend in section \ref{sec:agg}, I test whether we get better results by just aggregating up some subcomponents. I run a grid search to select up to 5 subcomponents that we forecast individually and then aggregate them up. The chosen variables are 'Shelter',  'Medical care services', 'Video and audio',
  'Communication'The models perform worse than the simple moving average models outlined above. Thus we conclude that it is not worth aggregating up the individual forecasts. 
\begin{figure}[h]
    \centering
    \includegraphics[width=1\textwidth]{bottomupaggregationmodelfoundusingcrossvalidation.pdf}
    \caption{Out of sample precdiction of a model that predicts inflation components individually and sums them up using basket weights. Using cross validation we try to select up to 5 components that we forecast individually. }
    \label{fig:bottomupaggregationmodelfoundusingcrossvalidation}
\end{figure}



Regarding the models presentend in section \ref{sec:eco}, I test whether we get better results by just using some of the economic varialbes outlined in table \ref{table:motivation}. I run a grid search to select up to 5 varialbes that can be used to enhance the MA 1, 7 model. 



describe the process to refine the algorithms and techniques, such as using cross-validation or changing the parameter settings.

\section{Results}
\subsection{Model Evaluation and Validation}
discuss the models and parameters used in the methodology. If no model is used, students can discuss the methodology using data visualizations and other means.
Chosen parameters


\subsection{Justification}
 discuss the final results in detail and explain why some models, parameters, or techniques perform better over others. Show and compare the results in tabular forms or charts.

\section{Conclusion}
\subsection{Reflection}
 summarize the end-to-end problem solution and discuss one or two particular aspects that you find interesting or difficult to implement.
\subsection{Improvement}
 provide suggestions for the next research to improve the experiment.

The problems of data availability and of data revisions are set aside in this investigation.
Thus, it is assumed that all series become available simultaneously in their final revised form.
In reality, however, some series are available only with a considerable lag, or are likely to be
revised significantly during the quarters that follow their initial publication. 9


\end{document}
